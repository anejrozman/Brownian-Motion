% Zadnja posodobitev: 14. 1. 2022
\documentclass[twoside,11pt]{article}
\usepackage[slovene]{babel}
\usepackage[utf8]{inputenc}
\usepackage{graphicx}
\usepackage[frame]{matrika}
\usepackage{mathtools}
\usepackage{epstopdf}
\usepackage{units}
% Po potrebi se lahk dodajo drugi standardni paketi, ki ne spreminjajo izgleda dokumenta

\begin{document}

\MAT{1}{11}{2024}
\naslov{Black–Scholesov model}

\avtor{Anej Rozman}

\institucija{Fakulteta za matematiko in fiziko \\ Univerza v Ljubljani}

\klasifikacija{~} 
\izvlecek{V članku so predstavljene evropske opcije, }
\title{Black–Scholes model}
\abstract{}

\glava\baselineskip=14.5pt

\smallskip

\section{Uvod}
    
    V svetu finančnih trgov, kjer so odločitve o vrednotenju ključnega pomena, je temeljen rezultat Black-Scholesov model. Razvit leta 1973 s strani Fischerja Blacka in Myrona Scholesa, ta matematični model predstavlja temeljno orodje pri ocenjevanju pravične vrednosti evropskih opcij. V tem članku predstavimo osnovne finančne koncepte, kot so osnovno premoženje, čas do zapadlosti, netvegana obrestna mera, volatilnost trga, itd. in motiviramo ter izpeljemo model kot limitni proces Cox-Ross-Rubinsteinovega (Binomskega) modela. Skozi to pokažemo, kako Black-Scholesova formula izhaja iz bolj preprostih finančnih modelov, hkrati pa ponuja naprednejši pristop k vrednotenju opcij, saj vse do danes ostaja najbolj poznan in pomemben zvezni model za vrednoenje. 

\section{Osnovni pojmi}
    V tem poglavju predstavimo finančne trge s ključnimi predpostvakami pod katerimi izvajamo vrednotenja ter evropske opcije kot izvedene finančne instrumente.

    ... predstavitev trgov, predpostavk, \dots

    \begin{definicija}
        \textit{Finančni instrument (ang. financial instrument)} je pogodba, ki vključuje kakršno koli finančno vrednost.
    \end{definicija}

    Najbolj poznani Finančni instrumenti so lastniški kot npr. delnica, nekateri pa imajo lastnost, da je njihova vrednost odvisna od vrednosti drugega finančnega instrumenta. Tem pravimo \textit{izvedeni finančni instrumenti (ang. derivative)}, instrumentom na katere so vezani pa pravimo \textit{osnovno premoženje (ang. underlying security/asset)}. Najbolj poznani primeri izvedenih finančnih instrumentov so opcije, vezane na delnice. 

    \begin{definicija}
        \textit{Evropska opcija (ang. European option)} je pogodba, ki daje imetniku pravico, ne pa tudi obveznosti, da kupi ali proda osnovno premoženje po dogovorjeni ceni na določen datum v prihodnosti, ki mu pravimo \textit{Čas zapadlosti (ang. expiration date)}.
    \end{definicija}

    \begin{zgled}
        Naj bo $S_0$ vrednost delnice ob času $t=0$. 


    \end{zgled}

        












\begin{thebibliography}{99}

\bibitem{1} S. Roman, \emph{Introduction to mathematics of finance : from risk management to options pricing}, Science \textbf{269} (2004), 238--275. 
\bibitem{2} W. Ketterle, D.M. Kurn, D.S. Durfee, N.J. van Druten, M.R. Andrews, M.-O. Mewes in K.B. Davis, \emph{Bose-Einstein Condensation in a Gas of Sodium Atoms}, Physics Review Letters \textbf{75} (1995), 3969--3973. 


\end{thebibliography}

\end{document}
